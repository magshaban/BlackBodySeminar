\documentclass[aspectratio=169]{beamer}
%\usepackage{hyperref}
\usepackage{graphicx}
\usepackage{amsmath}

%\usepackage[latin1]{inputenc}
\usepackage{multicol}

\newcommand{\highlight}[1]{\textbf{#1}}

\usetheme[progressbar=frametitle, numbering=none]{metropolis}
\setbeamercolor{background canvas}{bg=white}
\title{Spectrum of Blackbody}
\institute{{\footnotesize Guided by V.H. Belvadi} \\ Yuvaraja's College, Mysuru}
\author{Sudheendra B.R. \\ Suhas P.K. \\ Shahsank K.K.}
\date{}

\begin{document}

\begin{frame}[noframenumbering]
	\titlepage
\end{frame}

%\begin{frame}[noframenumbering]{Overview}
%	\begin{multicols}{2}
%		\tableofcontents[sections={1}] 
%		\columnbreak 								this did not work
%		\tableofcontents[sections={2-3}]
%	\end{multicols}
%\end{frame}

\begin{frame}[noframenumbering]{Overview}
\tableofcontents
\end{frame}

\metroset{numbering = counter}

\section{Part 1}

\subsection{Introduction to the Black Body problem}

\begin{frame}{Introduction}

{\large Ideally, any object which \highlight{absorbs and emits} all the radiation which falls on it, is considered to be a \highlight{Blackbody}.}

\end{frame}

\subsection{Charged oscillators}
 
\subsection{Rayleigh intensity function}

\begin{frame}{Rayleigh'law}

	\begin{itemize}
		
		\item Consider gaseous medium, in which we have a charged oscilator. \newline
		\item Due to random motion of the gaseous atoms, it often collides with the oscillator. \pause \newline
		\item As a result the gaseous atoms imparts some of it's kinetic energy to the oscillator.
		
	\end{itemize}
	
\end{frame}


\begin{frame}
	
	\begin{itemize}
	
		\item Mean kinetic energy acquired by the oscillator is, $\frac{1}{2}\,kT$. \newline
		\item And the total kinetic energy will be $ kT $. 
		
		\end{itemize}
		
\end{frame}


\begin{frame}
	\begin{itemize}

		\item We have a charge which has some kinetic energy, \pause \newline
		\item Means it \highlight{radiates} some of it's energy. \newline 
		\item Since the charge is radiating energy to infinity, \highlight{the system cannot attain equilibrium}.
			
	\end{itemize}
\end{frame}

	
\begin{frame}
	\begin{itemize}
			
		\item Since the charge is radiating energy to infinity,the system cannot attain equilibrium. \newline
		\item How do we tackle this situation? \pause \newline
		\item We confine the system.
		
	\end{itemize}
	
\end{frame}
	
		
\begin{frame}

	\begin{itemize}
			
		\item A charged oscillator surrounded by few gaseous atoms$^\dag$ placed inside a box. \newline
		\item Inner walls of the box is totally reflective.\newline
		\item Now, after some time we can claim that the system will attain \highlight{thermal equilibrium}
		%thought of adding the aprox. in foot note 
	\end{itemize}
	
\end{frame}
		
		
\begin{frame}
		
		So, the energy radiated  by the oscillator  per second can be given by, \newline
		{\large \[ \frac{1}{Q} = \frac{\left(\frac{dW}{dt}\right)}{\omega_0 W}\]}\newline
			{\small $Q$ is the radiation reaction. \\ $\omega_0$ is the natural frequency of vibration. \\ $W$ is the total energy content of the oscillator.} 
				
\end{frame}
	
		
\begin{frame}
		Also, we have 
		\[ \frac{1}{Q}= \frac{1}{\gamma} \] \newline
		{\small Where $\gamma$ is the dampping constant.} 
		\newline
		
		Remember the mean energy of the oscillator is $kT$
		
\end{frame}


\begin{frame}

	 \begin{center}
	 
	 	{\large Average energy loss due to radiation is given by,
	 	\[ \boxed{\left<\frac{dW}{dt}\right> = \gamma kT}\]	}
	 		
	 \end{center} 

\end{frame}
		
\begin{frame}
	
	\begin{itemize}
		
		\item Let, $ I(\omega)d\omega $ be the incident light energy with in a frequeny range $d\omega$. \newline
		\item Assuming all the radiation which falls on a \textit{cross section} $\sigma_s$ is absorbed completely, \newline
		\item Scattered light must be the product of $ I(\omega)d\omega $ and $ \sigma_s $. 

	\end{itemize}
	
\end{frame}
			
\begin{frame}
	
	{\large \[ \sigma_s = \frac{8\pi {r_0}^{2}}{3} \left( \frac{\omega^{4}}{(\omega^2-{\omega_0}^{2})^2+\gamma^2 \omega^2}\right) \]}\newline
		\begin{center}
			When $ \omega\,=\,\omega_0 $ \newline 
		\end{center}		
	{\large \[ \sigma_s = \frac{2\pi {r_0}^{2}{3} \omega_0^{2}}{3(\omega-\omega_0)^2+\dfrac{\gamma^2}{4}} \]}
	
\end{frame}
				
\subsection{The Ultra-Violet catastropy} 

\subsection{Plank's emphirical formula}
\begin{frame}{}
	\begin{itemize}

		\item Max Planck came up with an idea. \newline
		\item We had the experimental result, and what we needed was a theoretical explanation for the curve. \newline
		\item Planck studied the curve and came up with a mathematical formula. \newline
	\end{itemize}
\end{frame}
\begin{frame}{}
	\begin{itemize}

		\item He made an assumption that the energy level of the harmonic oscillator is quantized.  \newline
		\item The harmonic oscillator can take up energies only in the multiples of $ \mathbf{\hbar \omega} $.  \newline
		\item The probability of 
	\end{itemize}
\end{frame}

\section{Part 2}

\subsection{Intensity of a Black Body spectrum and the problem}

\begin{frame}{}
	\begin{itemize}

		\item The intensity of radiation for frequency $\omega$ is given by \[I(\omega)d\omega = \frac{\hbar \omega^3 d\omega}{ \pi^2 c^2 \left(e^{\frac{\hbar \omega}{kT}}-1 \right) } \] \newline
		\item Planck considered that the matter was quantized and not light. \newline
		\item Einstein came up with an alternate idea to solve Black body problem. \newline  
	\end{itemize}
\end{frame}

\subsection{Einstein's work on Black Bodies}

\begin{frame}{}
	\begin{itemize}

		\item Einstein considered that even the light was quantized, by taking that light is actually photons with energy $\hbar \omega$   \newline
		\item a pic is needed. \newline

	\end{itemize}
\end{frame}

\begin{frame}{}
	\begin{itemize}
		\item Einstein assumed that Planck's formula was right. \newline
		\item Consider two energy levels , say , the $n^{th}$ level and $m^{th}$ level. \newline
		\item A light of certain frequency is incident on atom and by absorbing a photon, transition from state n to state m occurs. 
 
	\end{itemize}
\end{frame}

\begin{frame}

	\begin{itemize}

		\item The probability of the transition is proportional to intensity of the light. \newline
		\item The proportionality constant is $B_{nm}$ and mathematically can be written as,
	\end{itemize}
		
			\begin{equation}
					 R_{n \to m} = N_{n} B_{nm} I(\omega) 
			\end{equation} 
						
\end{frame}

\section{Part 3}

\begin{frame}{title}
this should have 3 expt
\end{frame}


\begin{frame}{Case A}
	\begin{itemize}
		\item Detector 1 is set to detect only $ \alpha $ particles and Dectector 2 is set to detect only oxygen atoms.\newline
		\item Probability amplitude of the scattering is given as $ f(\theta)$ when they are at an angle $\theta$.\newline
		\item The probability of this event = $ \,\Bigr\rvert\,f(\theta) \,\Bigr\rvert\,^{2} $
	\end{itemize}		
\end{frame}


\begin{frame}{Case B}
	\begin{itemize}
		\item Set up the dectectors such that the detectors would dectect either $\alpha$ particle or oxygen atom.\newline
		\item We will not both to distinguish which particle is which one.\newline
		\item This means that if oxygen atom in position $\theta$ , then
		$\alpha$ particle on the opposite side is at an angle $\pi-\theta$.
	\end{itemize}
\end{frame}

\begin{frame}
	\begin{itemize}
		\item Probability amplitude of oxygen atom = $f(\pi-\theta)$ \newline
		\item Probability amplitude of $\alpha$ particle = $f(\theta)$ \newline
		\item The probability of $\alpha$  particle 
		being detected at dectector 1 = $\,\Bigr\rvert\,f(\theta)+f(\pi-\theta)\,\Bigr\rvert\,^{2}$
		
 	\end{itemize}
\end{frame}

\begin{frame}
	\begin{itemize}
		\item If  $\theta = \dfrac{\pi}{2}$,  then applying this to the expression  $\,\Bigr\rvert\,f(\theta)+f(\pi-\theta)\,\Bigr\rvert\,^{2}$. \newline
		\item Probability = $4 \,\Bigr\rvert\,f\left(\dfrac{\pi}{2}\right) \,\Bigr\rvert\,^{2}$ , if the particle interfere. \newline
		\item Probability = $2 \,\Bigr\rvert\, f\left(\dfrac{\pi}{2}\right) \,\Bigr\rvert ^{2}$ , if the particle donot interfere.\newline
		\item Add the amplitudes for the alternative process in which two particles simply exchange roles and there is an interference. \newline
	\end{itemize}
\end{frame}

\begin{frame}
	\begin{itemize}
		\item \textit{Can we apply the same logicto the electron-electron scattering ? }\newline
		\item \textit{OBSERVATION} : " When we have situation in which the identity of the electron which is arriving at a point is echanged with anothe one , the new amplitude interfere with old one with an opposite phase." \newline
		\item In electrons case , the interfering amplitude for exchange interfere with a negative sign. \newline
		Probability for electron = $\,\Bigr\rvert\,f(\theta)-f(\pi-\theta)\,\Bigr\rvert\,^{2}$
		 \end{itemize}
\end{frame}

\begin{frame}

	SPIN PROBABILITY TABLE
\end{frame}

\begin{frame}
	\frametitle{Identical Particles}
	
		{\large \textbf{Identical particles}\,, also called \textbf{indistinguishable particle}\,, are particles that cannot be distinguished from one another.}
		
\end{frame}

\begin{frame}{Identical - Indistinguishable Particles}
	\begin{itemize}
		\item Consider particle 'a' and particle 'b'. \newline
		\item Let the two particle collide and get sactttered in two different directions say '1' and '2' over a surface element $ds_{1}$ and $ds_{2}$ of the detector respectively. \newline
		\item If the particles are indistinguishable then the amplitudes of these process will add up. \newline
		\item Probability that the two particles arrive at $ds_{1}$ and $ds_{2}$ is \newline
		$\,\Bigr\rvert\,<1|a><2|b> + <2|a><1|b>\,\Bigr\rvert\,^{2} ds_{1} ds_{2}$
		
	\end{itemize}
\end{frame}

\begin{frame}
	\begin{itemize}
		\item Integrating over the area of the detector ,  if we let $ds_{1}$ and $ds_{2}$ rangr over the whole area $(\vartriangle S)$ , we could count each part of the area twice since the expression $\,\Bigr\rvert\,<1|a><2|b> + <2|a><1|b>\,\Bigr\rvert\,^{2} ds_{1} ds_{2}$ contains everything that can happen with any pair of surface elements $ds_{1}$ and $ds_{2}$. \newline
		\item $ Probability_{BOSE} = \dfrac{ \left(4\,\Bigr\rvert a \,\Bigr\rvert^{2} \,\Bigr\rvert b \,\Bigr\rvert^{2}\right)}{2} \left(\vartriangle S\right)$ \newline
		\item This is just twice what we got the probability for distinguished particles. 

	\end{itemize}
\end{frame}

\begin{frame}{State with n Bose particle}
	\begin{itemize}
		\item Consider n particles say a, b, c... scattered in n direction say 1, 2, 3 ... \newline
		\item Probability that each particle acting alone would go into an element of the surface ds of the detector is $\,\Bigr\rvert < >\,\Bigr\rvert^{2} ds$. 
	\end{itemize}
\end{frame} 

\begin{frame}
	\begin{itemize}
		\item \textbf{Assumption} :All particles are distinguishable. \newline
		\item Probability that n particles will be counted together in n different surface elements = $\,\Bigr\rvert a_{1}b_{2}c_{3}... \,\Bigr\rvert^{2}ds_{1}ds_{2}...$ \newline
		\item If the amplitude does not depend on where ds is located in the detector , then the \newline
		$Probability = \left(\,\Bigr\rvert a \,\Bigr\rvert^{2}\,\Bigr\rvert b \,\Bigr\rvert^{2}...\right)(ds_{1}ds_{2}...)$
	\end{itemize}
\end{frame} 

\begin{frame}
	\begin{itemize}
		\item Integrating each dS over the surface $\vartriangle S$ of the dectector \newline 
		$(P_{n})_{different} = \left(\,\Bigr\rvert a \,\Bigr\rvert^{2}\,\Bigr\rvert b \,\Bigr\rvert^{2}...\right)(\vartriangle S)^{n}$ \newline
		\item Now suppose that all the particle are Bose particles.
		\item For n particles, there are n! different , but indistinguishable  possibilities for which we must add the amplitudes. \newline
	\end{itemize}
\end{frame}

\begin{frame}
	\begin{itemize}
	\item Probability that n particles will be counted on the n surface elements is given by \newline $Probability = \left(\,\Bigr\rvert a_{1}b_{2}c_{3}... + a_{1}b_{3}c_{2}... \,\Bigr\rvert^{2}\right)(ds_{1}ds_{2}...)$
	\item $Probability = \left(\,\Bigr\rvert n!abc... \,\Bigr\rvert^{2}\right)(ds_{1}ds_{2}...)$
	\item Integrate each ds over the area $\vartriangle S$ of the detector \newline
	$(P_{n})_{BOSE} = n!\left(\,\Bigr\rvert abc... \,\Bigr\rvert^{2}\right)(\vartriangle S)^{n}$
	\end{itemize}
\end{frame}

\begin{frame}
	\begin{itemize}
	\item Compareing the probability when the particles are distinguishable and indistinguishable \newline
	$(P_{n})_{BOSE} = n!\left(\,\Bigr\rvert abc... \,\Bigr\rvert^{2}\right)(\vartriangle S)^{n}$ \newline
	$(P_{n})_{different} = \left(\,\Bigr\rvert a \,\Bigr\rvert^{2}\,\Bigr\rvert b \,\Bigr\rvert^{2}...\right)(\vartriangle S)^{n}$ 
	\item $(P_{n})_{BOSE} = n!(P_{n})_{different}$
	\end{itemize}
\end{frame}

\begin{frame}
	\begin{itemize}
		\item What is the probability that a Bose particle will go into particular state when there are already n particles present ?
	\end{itemize}
\end{frame}

\begin{frame}{Emission and Absorption of photons}
	\begin{itemize}
		\item When the light is emitted, a photon is "created".\newline
		\item Consider that there are some atom emitting n photons. \newline
		\item \textit{OBSERVATION} : The probability that an atom will wmit a photon into a particular final state is increased by the factor (n+1) if there are already n photons in that state.
	\end{itemize}
\end{frame}

\begin{frame}
\frametitle{The Blackbody Spectrum}
\end{frame}

\subsection{Amplitude-based descriptiom of identical particles}

\subsection{n Boson systems}

\subsection{Probability based emission and absorption of Black Bodies}

\end{document}


