\documentclass[aspectratio=169]{beamer}
\usepackage{hyperref}
\usepackage{graphicx}
\usepackage{amsmath}

\usepackage[latin1]{inputenc}
\usepackage{multicol}

\newcommand{\highlight}[1]{\textbf{#1}}

\usetheme[progressbar=frametitle, numbering=none]{metropolis}
\setbeamercolor{background canvas}{bg=white}
\title{Spectrum of Blackbody}
\institute{{\footnotesize Guided by V.H. Belvadi} \\ Yuvaraja's College, Mysuru}
\author{Sudheendra B.R. \\ Suhas P.K. \\ Shahsank K.K.}
\date{}

\begin{document}

\begin{frame}[noframenumbering]
	\titlepage
\end{frame}

\begin{frame}[noframenumbering]{Overview}
	\begin{multicols}{2}
		\tableofcontents[sections={1}]

		\columnbreak

		\tableofcontents[sections={2-3}]
	\end{multicols}
\end{frame}

\metroset{numbering = counter}
\section{Part 1}
\subsection{Introduction to the Black Body problem}

\subsection{Charged oscillators}
 
\subsection{Rayleigh intensity function}
 
\subsection{The Ultra-Violet catastropy} 

\subsection{Plank's emphirical formula}

\section{Part 2}
\subsection{Intensity of a Black Body spectrum and the problem}

\subsection{Einstein's work on Black Bodies}

\section{Part 3}
\subsection{Amplitude-based descriptiom of identical particles}

\subsection{n Boson systems}

\subsection{Probability based emission and absorption of Black Bodies}

\end{document}